\pagenumbering{roman} 
\setcounter{page}{1} 
\pagestyle{plain}

%%%%%%%%%%%%%%%%
%%% CREDITOS %%%
%%%%%%%%%%%%%%%%
\chapter*{Copyright}

\vspace{1cm}

\begin{figure}[ht]
    \centering
	\includegraphics[scale=1]{images/license.png}
\end{figure}

Esta obra está sujeta a una licencia de Reconocimiento -  NoComercial - SinObraDerivada

\href{https://creativecommons.org/licenses/by-nc-nd/3.0/es/}{3.0 España de CreativeCommons}.

%%%%%%%%%%%%%
%%% FICHA %%%
%%%%%%%%%%%%%
\chapter*{FICHA DEL TRABAJO FINAL}

\begin{table}[ht]
	\centering{}
	\renewcommand{\arraystretch}{2}
	\begin{tabular}{r | l}
		\hline
		Título del trabajo: & Detección de anomalías en el entorno del Internet de las cosas\\
		\hline
        Nombre del autor: & Gonzalo Pedro Mellizo-Soto Díaz\\
		\hline
        Nombre del colaborador/a docente: & Carlos Hernández Gañán\\
		\hline
        Nombre del PRA: & Jordi Casas Roma\\
		\hline
        Fecha de entrega (mm/aaaa): & 06/2019\\
		\hline
        Titulación o programa: & Máster Universitario en Ciencia de Datos\\
		\hline
        Área del Trabajo Final: & Minería de datos y Machine Learning \\
		\hline
        Idioma del trabajo: & Español\\
		\hline
        Palabras clave & Machine Learning, IOT, Anomaly Detection\\
		\hline
	\end{tabular}
\end{table}

%%%%%%%%%%%%%%%%%%%
%%% DEDICATORIA %%%
%%%%%%%%%%%%%%%%%%%
\chapter*{Cita}

\setlength\epigraphwidth{.8\textwidth}
\setlength\epigraphrule{0pt}

\epigraph{\itshape``Nuestro lema es: más humanos que los humanos"}{Eldon Tyrell, \textit{Blade Runner}}

%%%%%%%%%%%%%%%%%%%
%%% Agradecimientos %%%
%%%%%%%%%%%%%%%%%%%
\chapter*{Agradecimientos}

TO BE DEFINED

Si se considera oportuno, mencionar a las personas, empresas o instituciones que hayan contribuido en la realización de este proyecto.

%%%%%%%%%%%%%%%%
%%% RESUMEN  %%%
%%%%%%%%%%%%%%%%
\chapter*{Abstract}
\addcontentsline{toc}{chapter}{Abstract}

\onehalfspacing

In recent years the amount of connected devices has greatly increased, with an increasing number of applications in the industry each day. This devices can be subject of attacks causing instability or data leaks that can be dangerous both for the users and the enterprises, in order to avoid or confront them, security and early detection are become a must in a connected world. The focus is the monitoring and detection of the attacks in Internet of Things devices using state of the art Machine Learning techniques. Models such as SVM, DBScan or Isolation Forests have been used and assembled in order to identify with a better accuracy when an attack is happening. With this assembly attack detection has increased up to 15\% comparing to traditional methods and individual model usages and times have been considerably reduced. An active use of Machine Learning models has shown a great improvement at anomaly detection by securing the devices and decreasing the reaction times when facing attacks.

\vspace{0.5cm}

Durante los últimos año se encuentra una creciente cantidad de dispositivos conectados entre sí, con cada vez más aplicaciones en la industria. Estos dispositivos pueden ser atacados y provocar inestabilidad o una fuga de datos, por lo tanto la protección y la pronta detección de ataques y/o anomalías es vital en un mundo cada vez más conectado. El objetivo es la monitorización y detección de estos ataques en dispositivos del \textit{Internet of Things} utilizando técnicas del estado del arte de Machine Learning, para su detección y poder así responder con una mayor rapidez a los ataques. Para la detección se han utilizado modelos estadísticos, como SVM, DBScan o Isolation Forests, que en su conjunto permitan identificar con mayor precisión cuando se está produciendo un ataque. El conjunto de la clusterización con la clasificación de punto anómalos muestra una mayor robustez, frente al uso individual de cada uno de los modelos aumentando la detección en hasta un 15\%. Se demuestra como el uso de los modelos permite proteger los dispositivos y mejorar la seguridad al disminuir los tiempos de reacción frente a los ataques.

\vspace{0.5cm}
\textbf{Palabras clave}: Machine Learning, IOT, Anomaly Detection